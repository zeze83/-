% Options for packages loaded elsewhere
\PassOptionsToPackage{unicode}{hyperref}
\PassOptionsToPackage{hyphens}{url}
%
\documentclass[
]{article}
\usepackage{amsmath,amssymb}
\usepackage{iftex}
\ifPDFTeX
  \usepackage[T1]{fontenc}
  \usepackage[utf8]{inputenc}
  \usepackage{textcomp} % provide euro and other symbols
\else % if luatex or xetex
  \usepackage{unicode-math} % this also loads fontspec
  \defaultfontfeatures{Scale=MatchLowercase}
  \defaultfontfeatures[\rmfamily]{Ligatures=TeX,Scale=1}
\fi
\usepackage{lmodern}
\ifPDFTeX\else
  % xetex/luatex font selection
\fi
% Use upquote if available, for straight quotes in verbatim environments
\IfFileExists{upquote.sty}{\usepackage{upquote}}{}
\IfFileExists{microtype.sty}{% use microtype if available
  \usepackage[]{microtype}
  \UseMicrotypeSet[protrusion]{basicmath} % disable protrusion for tt fonts
}{}
\makeatletter
\@ifundefined{KOMAClassName}{% if non-KOMA class
  \IfFileExists{parskip.sty}{%
    \usepackage{parskip}
  }{% else
    \setlength{\parindent}{0pt}
    \setlength{\parskip}{6pt plus 2pt minus 1pt}}
}{% if KOMA class
  \KOMAoptions{parskip=half}}
\makeatother
\usepackage{xcolor}
\usepackage[margin=1in]{geometry}
\usepackage{longtable,booktabs,array}
\usepackage{calc} % for calculating minipage widths
% Correct order of tables after \paragraph or \subparagraph
\usepackage{etoolbox}
\makeatletter
\patchcmd\longtable{\par}{\if@noskipsec\mbox{}\fi\par}{}{}
\makeatother
% Allow footnotes in longtable head/foot
\IfFileExists{footnotehyper.sty}{\usepackage{footnotehyper}}{\usepackage{footnote}}
\makesavenoteenv{longtable}
\usepackage{graphicx}
\makeatletter
\def\maxwidth{\ifdim\Gin@nat@width>\linewidth\linewidth\else\Gin@nat@width\fi}
\def\maxheight{\ifdim\Gin@nat@height>\textheight\textheight\else\Gin@nat@height\fi}
\makeatother
% Scale images if necessary, so that they will not overflow the page
% margins by default, and it is still possible to overwrite the defaults
% using explicit options in \includegraphics[width, height, ...]{}
\setkeys{Gin}{width=\maxwidth,height=\maxheight,keepaspectratio}
% Set default figure placement to htbp
\makeatletter
\def\fps@figure{htbp}
\makeatother
\setlength{\emergencystretch}{3em} % prevent overfull lines
\providecommand{\tightlist}{%
  \setlength{\itemsep}{0pt}\setlength{\parskip}{0pt}}
\setcounter{secnumdepth}{-\maxdimen} % remove section numbering
\ifLuaTeX
  \usepackage{selnolig}  % disable illegal ligatures
\fi
\IfFileExists{bookmark.sty}{\usepackage{bookmark}}{\usepackage{hyperref}}
\IfFileExists{xurl.sty}{\usepackage{xurl}}{} % add URL line breaks if available
\urlstyle{same}
\hypersetup{
  hidelinks,
  pdfcreator={LaTeX via pandoc}}

\author{}
\date{\vspace{-2.5em}}

\begin{document}

\hypertarget{proposal}{%
\section{Proposal}\label{proposal}}

Group member: Yimeng Cai (yc3577), Zheshu Jiang (zj2379), Ze Li
(zl2746), Qianying Wu (qw2418)

\begin{enumerate}
\def\labelenumi{\arabic{enumi}.}
\tightlist
\item
  Project title:
\end{enumerate}

Regression Analysis on Socio-economic factors and Bioindicators of Type
II Diabetes by using R

\begin{enumerate}
\def\labelenumi{\arabic{enumi}.}
\setcounter{enumi}{1}
\tightlist
\item
  Project Motivation
\end{enumerate}

Type II Diabetes is a global health concern that affects more than 37
million Americans and places a significant burden on healthcare systems.
As a multifaceted disease, Type II diabetes was influenced by both
socio-economic factors and biological indicators. By analyzing a
comprehensive dataset from the Behavioral Risk Factor Surveillance
System, we aim to uncover the correlations between diabetes prevalence
and factors such as sex, education, income, lifestyle habits, and
bioindicators like blood glucose and cholesterol levels. Through
visualizations and statistical analysis in R, we aspire to highlight
patterns that could inform better prevention and management strategies,
thereby contributing to the broader dialogue on public health and
socio-economic policies related to diabetes etc..

\begin{enumerate}
\def\labelenumi{\arabic{enumi}.}
\setcounter{enumi}{2}
\tightlist
\item
  Data resources:
\end{enumerate}

\url{https://www.kaggle.com/datasets/alexteboul/diabetes-health-indicators-dataset/data}

\begin{itemize}
\item
  Data Description:

  The dataset was collected by the Behavioral Risk Factor Surveillance
  System (BRFSS) in 2015. This original dataset contains responses from
  441,455 individuals and has 330 features. For this project, we use the
  one with 253,680 responses and a three-class diabetes target variable
  indicating no diabetes, prediabetes, or diabetes. The other potential
  factors for diabetes are blood pressure, cholesterol, cholesterol
  check, BMI, stroke, heart disease, and habits of smoking, physical
  exercise, eating fruit.
\end{itemize}

\begin{enumerate}
\def\labelenumi{\arabic{enumi}.}
\setcounter{enumi}{3}
\tightlist
\item
  Intended Final Products
\end{enumerate}

\begin{itemize}
\item
  Demographics Visualization:

  From scratch, to understand the basic demographics information around
  type II diabetes, we decide to perform data visualization of type II
  diabetes occurrences among population in different biological sex,
  educational level, and income level.
\item
  Socio-economic Factors Analysis:

  In order to study socio - economic factors related to type II diabetes
  occurrences, we aim to distinguish the following three factors -
  smoking status, alcohol consumption, and body mass index (BMI). By
  implementing data analysis, statistical tests, along with data
  visualizations, we are able to study the correlation between each
  single factor and frequencies of type II diabetes.
\item
  Bioindicators Association:

  With the understanding of factors correlated with the disease, we plan
  to study the two bioindicators: blood pressure and cholesterol level
  associated with type II diabetes in order to decide which biological
  indicator is able to perform as a significant biomarker through
  statistical tests.
\end{itemize}

\begin{enumerate}
\def\labelenumi{\arabic{enumi}.}
\setcounter{enumi}{4}
\tightlist
\item
  Data Analysis
\end{enumerate}

\begin{itemize}
\item
  In order to make demographics visualization,we plan to use filter and
  group samples according to different factors of sex, education, and
  income and then visualize the frequencies and distributions for each
  factors through histograms, box plots etc.
\item
  To analyse the three socio-economic factors (sex, education, income),
  we are going to do sample t test, anova test, linear regression model,
  and model fitting to test for the associations and significance by
  looking at P-value, test statistics, or confidence intervals.
\item
  To analyse the two bioindicators, we will perform a hypothesis test
  along with statistical test like sample t test, multiple linear
  regression and comparing the P-value or test statistics to ensure the
  significance.
\end{itemize}

\begin{enumerate}
\def\labelenumi{\arabic{enumi}.}
\setcounter{enumi}{5}
\tightlist
\item
  Visualizations
\end{enumerate}

\begin{itemize}
\item
  We will use a stacked bar chart to show the proportion of individuals
  with heart disease within diabetic, pre-diabetic, and non-diabetic
  groups using ggplot.
\item
  To explore the associations between blood glucose and diabetes
  association, we plan to visualize with a histogram or density plot,
  overlaying the distribution of blood glucose levels for diabetic,
  pre-diabetic, and non-diabetic groups using ggplot. We will also
  create a scatter plot with a trend line to observe the relationship
  between blood glucose levels and diabetes by ggplot.
\item
  To visualize how diabetes prevalence varies across different
  demographic groups---such as sex, income, age, smoking status, and
  alcohol consumption---we will create a series of density plots. Each
  plot will display the distribution of diabetes cases within each
  demographic category, providing a clear visual comparison of how the
  disease's occurrence differs among various segments of the population.
\item
  Lastly, we will display side-by-side boxplots to show the spread of
  BMI across the dataset to give BMI Level Distribution by ggplot.
\end{itemize}

\begin{enumerate}
\def\labelenumi{\arabic{enumi}.}
\setcounter{enumi}{6}
\tightlist
\item
  Coding challenges
\end{enumerate}

\begin{itemize}
\item
  It could be challenging to format data suitable for plotting,
  including handling missing values, outliers, and categorization.
\item
  It could also be challenging to ensure accurate representation of data
  while constructing bar charts, dot plots, and density plots, ensuring
  accurate representation of data, for example, picking the right
  variables.
\item
  We also need to customize aesthetics of the graphs by mastering
  ggplot2's aesthetic parameters to differentiate categories visually by
  color, shape, or size. In addition, embedding statistical test
  results, like p-values, into plots and plotting dimensions and scales
  for clarity and to accommodate various factor levels could be
  difficult.
\end{itemize}

\begin{enumerate}
\def\labelenumi{\arabic{enumi}.}
\setcounter{enumi}{7}
\tightlist
\item
  Planned Timeline
\end{enumerate}

\begin{longtable}[]{@{}lll@{}}
\toprule\noalign{}
Timeline & Requirement & Predicted Timeline \\
\midrule\noalign{}
\endhead
\bottomrule\noalign{}
\endlastfoot
11/13-17 & Project review meeting & Nov 15th or 16th \\
12/9 & Report & Nov 24 \\
12/9 & webpage and screencast & Dec 1-7 \\
12/9 & peer assessment & Dec 7-8 \\
12/14 & In class discussion & \\
\end{longtable}

\end{document}
